\documentclass[UTF8]{article}
\usepackage{ctex}
\usepackage[colorlinks]{hyperref}
\usepackage{mathtools}
\usepackage{amsmath}
\usepackage{amssymb}
\usepackage{yhmath} 
\usepackage{wasysym}
\usepackage{indentfirst}
\usepackage{siunitx}
\usepackage{wasysym}
\usepackage{framed}
\usepackage[dvipsnames,svgnames]{xcolor}
\usepackage{tcolorbox}
\usepackage[strict]{changepage}
\usepackage{tikz}
\usepackage{xcolor}
\title{尼尔森-施赖埃尔定理}	

\begin{document}
	\maketitle
	\tableofcontents
	\newpage
	\section{尼尔森-施赖埃尔定理}
	
	\subsection{定义}
	设$S$是群$G$的子群,$S$在$G$中的一个\textbf{陪集代表系}$l$指的是$G$的一个如下定义的子集,它由每个陪集$Sb$中恰好取出一个元素$\ell(Sb) \in Sb$组成,并且满足$\ell(S) = 1$ \footnote{就像同余类一样,对$\mod n$都有集合$\{[1],[2],\cdots,[n-1]\}$,其中$1,2,\cdots,n-1$是代表元,也就是定义中的$b$}
	
	设$F$是以$X$为基的自由群,再设$S$是$F$的子群,给定$S$在$F$中的一个陪集代表系$\ell$,则对每个$x\in X$,$\ell(Sb)x$和$\ell(Sbx)$都在陪集$Sbx$中,从而
	$$
	t_{Sb,x} = \ell(Sb)x\ell(Sbx)^{-1}
	$$
	也在$S$中,可能有任意的$s_1\in S$和$s_2\in S$,使得$s_1bx * (s_2bx)^{-1} = s_1s^{-1}_2\in S$
	
	所以,不等于1的一切$t_{Sb,x}$的集合组成$S$的一组基。可知$S$是自由的。
	
	现在设$l$是自由群$F$的子群$S$的一个陪集代表系,元素$t_{Sb,x}$如上。定义$Y$为符号$y_{Sb,x}$上的自由群。从而$y_{Sb,x} \mapsto t_{Sb,x}$是双射,定义$\varphi :Y \to S$为
	$$
	\varphi :y_{Sb,x} \mapsto t_{Sb,x} = \ell(Sb)x\ell(Sbx)^{-1}
	$$
	的同态。 
	
	现在先定义陪集函数$F\to Y$。每个陪集$Sb$对应一个陪集函数。记为$u\mapsto u^{Sb}$。这些函数不是同态,我们对$\mid u\mid \geq 0$用归纳法同时定义他们,其中$u$是$X$上的约化字。对所有$x\in X$和所有陪集$Sb$,定义
	
	$$
	1^{Sb} = 1, x^{Sb} = y_{Sb,x}, (x^{-1})^{Sb} = (x^{Sbx^{-1}})^{-1}
	$$
	若$u = x^\epsilon v$是长度为$n+1$的约化字,其中$\epsilon = \pm 1$且$\mid v\mid =n$,定义
	$$
	u^{Sb} = (x^\epsilon)^{Sb}v^{Sbx^
	\epsilon}
	$$
	
	\newpage
	\subsection{引理}
	
	\begin{enumerate}
		\item 对所有$u,v\in F$,陪集函数满足$(uv)^{Sb} = u^{Sb}v^{Sbu}$
		\item 对所有$u\in F$,$(u^{-1})^{Sb} = (u^{Sbu^{-1}})^{-1}$
		\item 若$\varphi :Y\to S$是定义如下的同态:$\varphi:y_{Sb,x}\mapsto t_{Sb,x} = \ell(Sb)x\ell(Sbx)^{-1}$,则对所有的$u\in F$,$\varphi(u^{Sb}) = \ell(Sb)u\ell(Sbu)^{-1}$
		\item 由$\theta:u \mapsto u^S$给定的函数$\theta:S\to Y$是同态,并且$\varphi\theta = 1_S$
	\end{enumerate}
	
	\paragraph{证明:}我们对$\mid u\mid $进行归纳,其中$u$是既约字。当$\mid u \mid = 0$的时候,$u =1$,则$(uv)^{Sb} = v^{Sb}$。对于等式右边,$1^{Sb}v^{Sb1} = v^{Sb}$
	
	对于归纳步骤,我们记$u = x^\epsilon w$,则
	$$
	\begin{aligned}
			(uv)^{Sb} =& (x^\epsilon)^{Sb}(wv)^{Sbx^\epsilon} & \text{陪集函数定义}\\
			=&(x^\epsilon)^{Sb} w^{Sbx^\epsilon}v^{Sbx^\epsilon w}& \text{由归纳假设得到}\\
			=&(x^\epsilon)^{Sb}w^{Sbx^\epsilon} v^{Sbu}\\
			=& (x^\epsilon w)^{Sb} v^{Sbu}\\
			=& u^{Sb}v^{Sbu}
	\end{aligned}
	$$
	
	对第二个命题,注意到
	$$
	1 = 1^{Sb} = (u^{-1}u)^{Sb} = (u^{-1})^{Sb} (u^{Sbu})^{-1}
	$$
	
	对第三个命题,注意$\varphi$确实是定义了一个同态,利用自由群的定义,$Y$是以所有$y_{Sb,x}$为基的自由群。我们依然是对$\mid u\mid \geq  0$做归纳。 
	
	首先,对$\varphi(1^{Sb}) = \varphi(1) = 1$,因为$\ell(S)1\ell(S1)^{-1} = 1$
	
	对归纳步骤,我们令$u = x^\epsilon v$,其中$u$是既约的,则
	$$ 
	\begin{aligned}
			\varphi(u^{Sb}) = \varphi((x^\epsilon v)^{Sb}) = &\varphi((x^\epsilon)^{Sb} v^{Sbx^\epsilon})\\
		=& \varphi((x^\epsilon)^{Sb})\varphi(v^{Sbx^\epsilon})\\
		=& \varphi\left((x^{\varepsilon})^{Sb}\right)\ell(Sb x^{\varepsilon}) v \ell(Sb x^{\varepsilon} v)^{-1}
	\end{aligned}
	$$
	
	最后一个等号来自归纳假设,那么对$\epsilon$,有两种情况,若$\epsilon = 1$,则
	$$
	\begin{aligned}
		\varphi(u^{Sb}) &= \ell(Sb) x \ell(Sb x)^{-1} \ell(Sb x) v \ell(Sb x v)^{-1} \\
		&= \ell(Sb) x v \ell(Sb x v)^{-1} \\
		&= \ell(Sb) u \ell(Sb u)^{-1}.
	\end{aligned}
	$$
	
	若$\epsilon = -1$,则
	
	$$
	\begin{aligned}
		\varphi(u^{Sb}) =\varphi(x^{-1}v) &=  \varphi((y_{Sbx^{-1},x})^{-1})\ell(Sbx^{-1})v\ell(Sbx^{-1}v)^{-1}\\
		&= \left(\ell(Sbx^{-1}) x \ell(Sbx^{-1}x)^{-1}\right)^{-1} \ell(Sbx^{-1}) v \ell(Sbx^{-1}v)^{-1} \\
		&= \ell(Sb) x^{-1} \ell(Sbx^{-1})^{-1} \ell(Sbx^{-1}) v \ell(Sbx^{-1}v)^{-1} \\
		&= \ell(Sb) x^{-1} v \ell(Sbx^{-1}v)^{-1} \\
		&= \ell(Sb) u \ell(Sbu)^{-1}.
	\end{aligned}
	$$
	
	对第四个命题,定义$\theta:S\to Y$为
	$$
	\theta:u\mapsto u^S
	$$
	$\theta$是$b=1$时,陪集函数$u\mapsto u^{Sb}$对$S$的限制。现在,若$u,v\in S$,则
	$$
	\theta(uv) = (uv)^S = u^Sv^{Su} = u^Sv^S = \theta(u)\theta(v)
	$$
	因为$Su = S$,其中$u\in S$,所以$\theta$是一个同态,更多的。若$u\in S$,则由第三个定理给出
	$$
	\varphi\theta(u) = \varphi(u^S) = \ell(S1)u\ell(S1u)^{-1} = u
	$$
	\subsection{推论}
	
	若$S$是自由群$F$的子群且$\ell$是$S$在$F$中的陪集代表系,则所有不等于1的$ t_{Sb,x}$生成$S$
	
	\paragraph{证明:}由于$\varphi\theta = 1_S$,函数$\varphi:Y \to S$是一个满射,因此Y的生成元$y_{Sb,x}$的像$t_{Sb,x}$生成$\text{im}\varphi = S$
	
	接着我们试着证明$S$有表现
	$$
	S = \{y_{Sb,x},\text{所有}x\in X,\text{所有陪集}Sb\mid \ell(Sb)^S,\text{所有陪集}Sb\}
	$$
	
	\newpage
	\subsection{引理}若$\ell$是$S$在$F$中的陪集代表系,则$\ker\varphi$是一切$\ell(Sb)^S$生成的$Y$的正规子群。
	
	\paragraph{证明:}令$N$是所有$\ell(Sb)^S$生成的$Y$的正规子群,再令$K = \ker\varphi$。由引理1.2的4,$\theta:S\to Y$是一个同态且$\varphi\theta = 1_S$,其中$\varphi:y_{Sb,x} \mapsto t_{Sb,x}$且$\theta:u\mapsto u^S$
	
	现在,引入命题
	
	\begin{framed}
		设 \( Y \) 和 \( S \) 是群,并设 \(\varphi : Y \rightarrow S\) 和 \(\theta : S \rightarrow Y\) 是同态满足 \(\varphi \theta = 1_S\)。
		
		(Ⅰ)如果 \(\rho : Y \rightarrow Y\) 定义为 \(\rho = \theta \varphi\),证明 \(\rho \rho = \rho\) 以及对每个 \(a \in \operatorname{im}\theta\), \(\rho(a) = a\)。(同态 \(\rho\) 叫做收缩。)
		
		(Ⅱ)如果 \(K\) 是 \(Y\) 的正规子群,它由一切 \(y^{-1} \rho(y)\) 生成,其中 \(y \in Y\),证明 \(K = \ker\varphi\)。
		
		\text{证明:}$\rho\rho = \theta\varphi\theta\varphi = \theta 1 \varphi = \rho$。现在,对$a\in \text{im}\theta$,设$a = \theta(b)$,就有$\rho(a) = \rho(\theta(b)) = \theta(\varphi(\theta(b))) = \theta(b) = a$
		
		\text{证明2:}现在$K\lhd Y$,其中$K = \{y^{-1}\rho(y):y\in Y\}$,那么$\varphi(k) = \varphi(y^{-1}\rho(y)) = \varphi(y^{-1} \theta\varphi(y^{-1})) = \varphi(y^{-1}) (\varphi\circ \theta\circ\varphi(y)) = \varphi(y^{-1} * y) = \varphi(1) = 1$,其中$k\in K$
		
		
	\end{framed}
	因此,由上述命题,$K$是由$\{y^{-1}\rho(y):y\in Y\}$生成的正规子群,其中$\rho = \theta\varphi$。现在,由引理1.2的1.就有
	
	$$
\begin{aligned}
	y_{Sb,x}^{-1}\rho(y_{Sb,x}) &= y_{Sb,x}^{-1}\left(\ell(Sb)x\ell(Sbx)^{-1}\right)^S \\
	&= y_{Sb,x}^{-1}\ell(Sb)^S x^{Sb}\left(\ell(Sbx)^{-1}\right)^{Sbx} \\
	&= \left( y_{Sb,x}^{-1}\ell(Sb)^Sy_{Sb,x}\right)\left(\ell(Sbx)^{-1}\right)^{Sbx},
\end{aligned}
	$$
	
	其中$x^{Sb} = y_{Sb,x}$是陪集函数$u\to u^{Sb}$定义中的一个等式。引理1.2的2给出$(\ell(Sbx)^{-1})^{Sbx} = (\ell(Sbx)^S)^{-1}$。则

	$$
	y_{Sb,x}^{-1}\rho(y_{Sb,x}) = y^{-1}_{Sb,x}\ell(Sb)^S y_{Sb,x}(\ell(Sbx)^S)^{-1}
	$$
	由上述等式,则$y^{-1}_{Sb,x}\rho(y_{Sb,x}) \in N$,从而$K\leq N$。因为正规子群对共轭封闭。$ y^{-1}_{Sb,x}\ell(Sb)^S y_{Sb,x}\in N$。现在,$N$是所有陪集生成的群,那么$\ell(Sbx)^S \in N$,由封闭性可知$((\ell{Sbx})^S)^{-1}$也在N中。所以$K \leq N$
	
	
	对于反包含,$\ell(Sb)^S\in K$当且仅当$\ell(Sbx)^S\in K$。
	
	现在,我们对$\mid \ell(Sb)\mid $归纳来证明另一个包含光系,当$\mid \ell(Sb)\mid =0 $的时候,就有$y^{-1}_{1,x}\rho(y_{1,x}) = 1\Rightarrow 1 = 1$
	
	剩下的以后再补
	
	\subsection{定义:施赖埃尔陪集代表系}
	
	我们设$F$是以$X$为基的自由群,并设$S$是$ F$的子群,一个\textbf{施赖埃尔陪集代表系}指的是满足下列性质的陪集代表系$\ell$,若$\ell(Sb) = x^{\epsilon_1}_1 x^{\epsilon_2}_2\cdots x_n^{\epsilon_n}$是既约的,则每个初始段$x_1^{\epsilon_1}\cdots x_k^{\epsilon_k}$也在陪集代表系中,其中$1\leq k \leq n$
	
	\subsection{引理}
	$F$的每个子群$S$都存在施赖埃尔陪集代表系
	
	\paragraph{证明:} 我们定义陪集$Sb$的长度为元素$sb\in Sb$的最小长度,记为$\mid Sb\mid $,现在对$\mid Sb\mid $用归纳法。我们要证明存在代表元$\ell(Sb)\in Sb$使得其对一切初始段都是长度较短的陪集的代表元。
	
	一开始定义$\ell(S) = 1$。现在,对归纳假设我们有$\mid Sz \mid = n+1$和$ux^\epsilon \in Sz$,其中$z  =\pm 1$和$\mid ux^\epsilon\mid = n+1$。若$\mid Su\mid = m < n$,那么就会有一个长度为$m$的代表元$v$,使得$vx^\epsilon$是$Sz$的代表元\footnote{考虑$H = \{(1),(12)\}$,陪集$(13)H = \{(13),(123)\}$,所以有两个代表元$(13)H,(123)H$,但他们是一样的,可以替换$(123)$为$(13)$}。但$vx^\epsilon<n+1$,所以$\mid Su\mid = n$。由归纳假设,存在$b=\ell(Su)$使得每个初始段也是代表元。于是定义$\ell(Sz) = bx^\epsilon$
	
	\subsection{尼尔森-施赖埃尔定理}
	
	自由群$F$的每个子群$S$都是自由的,实际上,若$X$是$F$的基,$\ell$是$S$在$F$中的施赖埃尔陪集代表系,则$S$的一个基由一切不等于1的$t_{Sb,x} = \ell(Sb)x\ell(Sbx)^{-1}$组成
	
	\newpage
	\paragraph{证明:} 对引理的1.2的$\varphi$应用第一同构定理,就有$S\cong Y/K$,其中$Y$是一切符号$y_{Sb,x}$为基的自由群,而$K = \ker\varphi$,利用引理$1.4$,现在$K$等价于一切$\ell(Sb)^S$生成的正规子群。利用下述习题
	
	
	\begin{framed}
		设$F$是以$X$为基的自由群且$A\subseteq X$,证明若$N$是$A$生成的$F$的正规子群,则$F/N$是自由群
	\end{framed}
	
	我们只需要证明$K$等于一切特定的$y_{Sb,x}$生成的$Y$的正规子群$T$。也就是满足$\varphi(y_{Sb,x}) = 1 = t_{Sb,x}$的那些$y_{Sb,x}$。显然的是,利用定义有$T\leq K = \ker\varphi$,现在我们来证明反包含。
	
	我们对$\mid \ell(Sv)\mid $用归纳法证明。若$\mid \ell(Sv)\mid = 0$,则$\mid \ell(Sv)\mid =  \ell(S)  = 1$。现在,若$\mid \ell(Sv)\mid > 0$,则$\ell(Sv) = ux^\epsilon$,其中$\epsilon = \pm 1$且$\mid u\mid < \mid \ell(Sv)\mid $。由于$\ell$是施赖埃尔陪集代表系,那么$u$是代表元,有$u = \ell(Su)$ 利用引理1.2的1,就有
	$$
		\ell(Sv)^S = u^S(x^\epsilon)^{Su}
	$$
	现在,归纳假设告诉我们$u^S$是特定$y_{Sb,x}$上的字,那么$u^S \in T$。
	
	剩下,我们证明$(x^\epsilon)^{Su}$是特定$y_{Sb,x}$上的字。那么现在有两种情况,$\epsilon = \pm 1$。那么$(x^\epsilon)^{Su} = x^{Su} = y_{Su,x}$。而$v = ux$且$\ell$是施赖埃尔陪集代表系,由定义,$v = ux $也在其中,所以$\ell(Sux) = ux$。从而
	
	$$
	\varphi(y_{Su,x}) = t_{Su,x} = \ell(Su)x\ell(Sux)^{-1} = ux(ux)^{-1} =1
	$$
	
	因此
	
	$$
	\varphi((x^{-1})^{Su}) = (t_{Sux^{-1},x})^{-1} = [\ell(Sux^{-1})x\ell(Sux^{-1}x)]^{-1} = [\ell(Sux^{-1})x\ell(Su)]^{-1}
	$$
	
	而$\ell$是施赖埃尔陪集代表系,就有$\ell(Su) = u$和$\ell(Sux^{-1}) = \ell(Sv) = v = ux^{-1}$。最后
	$$
	\varphi((x^{-1})^{Su}) = [(ux^{-1})xu^{-1}]^{-1} = 1
	$$
	所以他的逆$y_{Sux^{-1}}$也是特定的$y_{Sb,x}$中的一个,从而$(x^{-1})^{Su}\in T$。证明完毕
	
	
	和阿贝尔群不同的是,有限生成群的子群未必是有限生成的
	
	\subsection{命题}
	若$F$是秩为2的自由群,则它的换位子群$F'$是秩为无限的自由群。
	
	\paragraph{证明:}我们设$\left<x,y \right>$是$F$的基,那么$F/F'$就是以$\{xF',yF'\}$为基的自由阿贝尔群。因此每个陪集$F'b$有唯一形如$x^my^n$的代表元,其中$m,n\in\mathbb{Z}$。由于每个$x^my^n$的子字具备同样形式的字(由交换性得到),所以我们选取$\ell(F'b) = x^my^n$的陪集代表系就是施赖埃尔代表系,若$n>0$,则$\ell(F'y^n) = y^n$,但是$\ell(F'y^nx) = xy^n \neq y^nx$,从而存在无穷多个这样的元素$y_{Sy^n,x} = \ell(F'y^n)x\ell(F'y^nx)^{-1}\neq 1$。由尼尔森-施赖埃尔定理证毕。
	
	但,即使是有限生成自由群的任一子群未必是有限生成的,但指数有限的子群必然是有限生成的
	
	\subsection{定理}
	若$F$是具备有限秩$n$的自由群,则$F$的每个具有有限指数$j$的子群$S$也是有限生成的,事实上,$\text{rank}(S) = jn-j+1$
	
	\paragraph{证明:} 我们设$X = \{x_1,\cdots,x_n\}$是$F$的基,再设$\ell = \{\ell(Sb)\}$是施赖埃尔陪集代表系。利用定理1.7,$S$的一组基是由那些不等于$1$的元素$t_{Sb,x}$组成的,其中$x\in X$,$Sb$有$j$种取法,而$x$有$n$种。所以$S$的基中至多有$jn$个元素,所以$rank(S)\leq jn$是有限生成的。
	
	定义平凡有序对$(Sb,x) = t_{Sb,x} = 1$。也就是$\ell(Sb)x = \ell(Sbx)$,我们现在要证明的是陪集族$(Sb\neq S)$和平凡有序对之间存在双射$\psi$,从而证明存在$j-1$个有序对。这样,只要剔除了$j-1$个1即可得到rank为
	$$
	\text{rank}(S) = jn - (j-1)
	$$
	由于$Sb\neq S$。那么$\ell(Sb) = ux^\epsilon$,因为$\ell$是施赖埃尔代表系,有$u\in \ell$,定义$\varphi(Sb)$如下
\[
\psi(Sux^\epsilon) = 
\begin{cases} 
	(Su, x) & \text{如果 } \epsilon = +1; \\ 
	(Sux^{-1}, x) & \text{如果 } \epsilon = -1. 
\end{cases}
\]

注意$\psi(Sux^\epsilon)$是平凡有序对,假设$\epsilon = 1$,则$\ell(Sux) = \ell(Sb) = b = ux$,从而$\ell(Su)x = ux$和$t_{Su,x} = 1$,注意$\ell(Su)x\ell(Sux)^{-1} = t_{Su,x} = 1$。若$\epsilon = -1$,则$\ell(Sbx) = \ell(Sux^{-1}x) = \ell(Su) = u$。从而$\ell(Sb)x = bx = ux^{-1}x = u$,就有$t_{Sb,x} = 1$

为了得到$\psi$是单射,设$\psi(Sb) = \psi(Sc)$,其中$b = ux^\epsilon$和$c = vy^\eta$。我们假设$x,y$在$F$的基中且$\epsilon = \pm 1$,$\eta = \pm 1$。那么就存在4种可能性,他们依赖于$\epsilon$和$\eta$

$$
(Su,x) = (Sv,y),~ (Su,x) = (Svy^{-1},y);~~(Sux^{-1},x) = (Sv,y),~~(Su,x) = (Svy^{-1},y)
$$

第一种情况,若$ (Su,x) = (Sv,y)$,则$Su= Sv$。那就有$Sb= Sux = Svx = Sc$。若$ (Su,x) = (Svy^{-1},y) $,那么$Su = Svx^{-1} = Sc$,从而$\ell(Su) = \ell(Sc) = c$,但$(Su,x)$是平凡有序对,那么$\ell(Su)x = \ell(Sux) = b$,从而$b = \ell(Su)x = cx = vx^{-1}x $,这是个矛盾。因为$b$是既约的。剩下的两种情况如上。最后,若$(Sux^{-1},x) = (Svx^{-1},x)$,则回到第一种情况,有$Sb = Sux^{-1} = Svx^{-} Sc$

最后,我们证明$\psi$是满射。取平凡有序对$(Sw,x)$,也就是$\ell(Sw)x = wx = \ell(Swx)$,现在,$w = ux^\epsilon$,其中$u\in \ell$,且$\epsilon = \pm 1$,若$\epsilon = 1$,则$w$不以$x^{-1}$结尾,从而$\psi(Swx) = (Sw,x)$。若$\epsilon = -1$则$w$以$x^{-1}$结尾,从而$\psi(Su) = (Sux^{-1}) = (Sw,x)$,实际上就是验证了一下定义。

\subsection{定理}

存在不同构的有限生成群$G$和$H$,他们的每一个同构于另一个的子群

\paragraph{证明:}若$G$是秩为2的自由群,$H$是秩为3的自由群,则$G\not\cong H$,现在,选择基里面任意2个元素生成的子群就是秩为2的子群,这表明$G$于此同构,最后,由于$G$的换位子群是无限群,它包含了一个秩为3的自由子群,即$H$和$G$的子群同构。
\end{document}